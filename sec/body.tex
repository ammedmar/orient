% !TEX root = ../lk_sq.tex

%\section*{The first Wu formula at the geometric cochain level} \label{s:statement}

Throughout this note $M$ denotes a (non-necessarily orientable) smooth $d$-manifold.

\subsection*{Preliminaries}

Let us consider the \textbf{Steenrod square} operations
\[
\Sq^i \colon H^m(X; \Ftwo) \to H^{m+i}(X; \Ftwo)
\]
defined for any space $X$.
Recall that $\Sq^0$ is the identity and that $\Sq^i[\alpha] = 0$ if $i < 0$ or $i > m$, where $m$ is the degree of $[\alpha]$.

The $i^\th$ \textbf{Wu class} $v_i \in H^i(M; \Ftwo)$ is defined by the identity
\[
\Sq^i [\alpha] = v_i \smallsmile [\alpha]
\]
holding for every $[\alpha]$.
Its existence is guaranteed by the non-degeneracy of the Poincar\'e duality pairing.
The $i^\th$ \textbf{Stiefel--Whitney class} $w_i \in H^i(M; \Ftwo)$ of (the tangent bundle of) $M$ is defined by the identity
\[
(w_0 + w_1 + \dotsb) = (\Sq^0 + \Sq^1 + \dotsb)(v_0 + v_1 + \dotsb).
\]
Explicitly,
\begin{align*}
	w_0 &= v_0 \\
	w_1 &= v_1 \\
	w_2 &= v_2 + Sq^1(v_1) \\
	& \ \, \vdots
\end{align*}

\subsection*{Cohomological relation}

The goal of this note is to \emph{geometrically} lift the following relation holding in mod 2 cohomology to the integral cochain level.

Let $\gamma \colon S^1 \to M$ be a smooth map to a closed $d$-manifold.
Let $\gamma^\ast w_1(TM)$ be the pullback of the first Stiefel--Whitney class of the tangent bundle of $M$ along $\gamma$.
Let $\gamma_\ast[S^1]$ be the pushforward of the mod 2 fundamental class of $S^1$ along $\gamma$ and $\dual \gamma_\ast[S^1]$ its Poincar\'e dual.
Notice that, tautologically,
\[
\gamma_\ast[S^1] = \dual \gamma_\ast[S^1] \smallfrown [M].
\]
Therefore,
\begin{equation} \label{e:cohomological_relation}
	\begin{split}
	\angles[\big]{\gamma^\ast w_1(TM), [S^1]} &=
	\angles[\big]{w_1(TM), \gamma_\ast [S^1]} \\ &=
	\angles[\big]{w_1(TM), \dual \gamma_\ast[S^1] \smallfrown [M]} \\ &=
	\angles[\big]{w_1(TM) \smallsmile \dual \gamma_\ast[S^1], [M]} \\ &=
	\angles[\big]{\Sq^1 \dual \gamma_\ast[S^1], [M]}.
	\end{split}
\end{equation}

\subsection*{Obstructions to sections}

Similarly to how the top Stiefel--Whitney class has an integral lift --the Euler class-- when $M$ is oriented, the odd Stiefel--Whitney classes have integral lifts using twisted coefficients.
In particular, we denote the integral lift of $w_1$ by $\widetilde w_1$.

\subsection*{Geometric cochains}

For a closed manifold $N$ denote by $C^*_{\Gamma}(M)$ the complex of geometric cochains introduced in \cite{medina2021flowing}.
Its degree $m$ part consists of linear combinations of equivalence classes of smooth maps to $N$ from co-oriented manifolds with corners of codimension $m$.
Its differential is defined by the geometric boundary of the generators.
It is a model for the integral cohomology of $M$.
In fact, we have the following comparison at the cochain level with the simplicial cochains $\cochains(X)$ of any triangulation $\bars{X} \to M$.
We first restrict to the subcomplex $C^*_{\Gamma \pitchfork X}(N)$ of geometric cochains that are transverse to the triangulation, claiming that the quasi-isomorphism type is preserved, and then define a quasi-isomorphism
\[
\cI \colon C^*_{\Gamma \pitchfork X}(N) \to \cochains(X)
\]
by counting the number of intersection with signs determined by comparing the co-orientation of the geometric cochain and a natural orientation of the standard simplices.

\subsection*{Whitney vector fields}

We will use a family considered by Whitney in \cite{whitney1940sphere_bundles} of vector fields $f_1, \dots, f_n$ on the standard $n$-simplex, for which we use the following topological model
\[
\gsimplex^n = \set[\big]{(x_0, \dots, x_n) \mid \textstyle \sum x_i = 1,\ x_i \geq 0}
\]
with canonical inclusions $\delta_i \colon \gsimplex^{n-1} \to \gsimplex^n$ given by
\[
\delta_i(x_0, \dots, x_{n-1}) = (x_0, \dots, x_{i-1}, 0, x_{i+1}, \dotsm, x_{n-1}).
\]
Using the canonical basis $e_0, \dots, e_n$ of $\R^{n+1}$ and the identification of this space with its tangent bundle at each point, we define for $j \in \{1, \dots, n\}$ the $j^\th$ \textit{Whitney vector field} $f_j$ by
\[
f_j(x) = \sum_{\mathclap{i_0 < \dots < i_j}} x_{i_0} \dotsm \, x_{i_j} (e_j - x).
\]
These vector fields are natural in the following sense.
For any canonical inclusion $\delta_i \colon \gsimplex^{n-1} \to \gsimplex^n$ and Whitney vector field $f_j$ with $j \in \{1, \dots, n-1\}$ we have:
\begin{equation} \label{e:pushforward}
	\delta_{i\ast} f_j =
	\begin{cases}
		f_j |_{\delta_i(\gsimplex^{n-1})} & \text{if } i > j, \\
		f_{j+1} |_{\delta_i(\gsimplex^{n-1})} & \text{if } i \leq j.
	\end{cases}
\end{equation}
That is to say, the pushforward of the $j^\th$ Whitney vector fields along the $i^\th$ canonical inclusion is equal to the restriction of the $j^\th$ or $(j+1)^\th$ Whitney vector field to the image of said inclusion depending on the order relation between $i$ and $j$.

\subsection*{Orientations}

For any positive integer $n$, the set $\set{f_1, \dots, f_n}$ is linearly independent in the interior of $\gsimplex^n$ and defines a orientation for which $f_1 \wedge \dots \wedge f_n$ is a positive.
We observe that by \cref{e:pushforward}, the pushforward along $\delta_i \colon \gsimplex^{n-1} \to \gsimplex^n$ satisfies
\[
\delta_{i \ast} (f_1 \wedge \dots \wedge f_{n-1}) =
\begin{cases}
	f_2 \wedge \dots \wedge f_n &
	\text{if } i = 0, \\
	f_1 \wedge \dots \wedge \widehat{f_i} \wedge \dots \wedge f_n
	& \text{if } i \neq 0.
\end{cases}
\]
For $i \neq 0$, the outer pointing normal on $\delta_i(\gsimplex^{n-1})$ can be represented by $- f_i$ and we have
\[
-f_i \wedge f_1 \wedge \dots \wedge \widehat{f_i} \wedge \dots \wedge f_n =
(-1)^i f_1 \wedge \dots \wedge f_n.
\]
The outward pointing normal for the case $i = 0$ is represented by $x$ and one can compute that
$x \wedge f_2 \wedge \dots \wedge f_n$ represents the same orientation as $f_1 \wedge \dots \wedge f_n$.

\begin{lemma}
	The orientation induced by the inclusion $\delta_i \colon \gsimplex^{n-1} \to \gsimplex^n$ and the outer pointing normal agrees with that in the target if and only if the integer $i$ is even.
\end{lemma}

\subsection*{First Stiefel--Whitney class}

Let us assume $\gamma \colon S^1 \to M$ is an embedding transverse to the triangulation $\bars{X} \to M$.
We further assume it is oriented and co-oriented.
That is to say, we have chosen and orientation of its tangent bundle, say represented by a vector field $v$, and an orientation of its normal bundle in $M$.

\begin{definition}
	Let $\omega_1 \subset S^1$ to be the locus where the set $\set{\gamma_\ast (v), f_1, \dots, f_{n-1}}$ fails to be linearly independent.
\end{definition}

As a further transversality assumption, we demand this locus to consist of isolated points.

\begin{proposition}
	The cohomology class represented by the geometric cocycle $\omega_1 \hookrightarrow S^1$ is equal to $\gamma^\ast \widetilde w_1(TM)$.
%	the first Stiefel-Whitney class of $\gamma^\ast TM$ with integer coefficients.
%	In particular, its mod 2 reduction is equal to $w_1(\gamma^\ast TM)$.
\end{proposition}

We remark that this statement fits well with the original definition of the first Stiefel--Whitney class $\widetilde w_1(TM)$, as the primary obstruction to the existence of $n$ linearly independent vector fields in $TM$.

%We will obtain this statement from a local lift to the cochain level of the identity

\subsection*{First Steenrod square}

Consider now the composition $W_1 \hookrightarrow S^1 \xra{\gamma} M$ as a geometric cocycle on $M$.

\begin{proposition}
	The first Steenrod square of the mod 2 cohomology class represented by the geometric cocycle $S^1 \xra{\gamma} M$ is represented by $\omega_1 \hookrightarrow S^1 \xra{\gamma} M$.
\end{proposition}

This result follows from a local cochain level statement we now present.
Let
\[
\smallsmile_i \colon \cochains(K; \Z) \otimes \cochains(K; \Z) \to \cochains(K; \Z)
\]
be Steenrod's cup-$i$ product introduced in \cite{steenrod1947products} and axiomatized in \cite{medina2022axiomatic}.
Recall that for any mod 2 class $[\alpha]$ with $\alpha \in \cochains(K)_{d-1}$ we have
\[
\Sq^1 [\alpha] = \big[ \alpha \smallsmile_{d-2} \alpha \big],
\]
which motivates the notation
\[
\SQ^1 \alpha \defeq \alpha \smallsmile_{d-2} \alpha.
\]
\begin{proposition}
	The following identity holds in $\cochains(X)$ if $\gamma$ does not visit a top simplex more than once:
	\begin{equation} \label{e:identity}
		\cI(W_1 \hookrightarrow S^1 \xra{\gamma} M) = \SQ^1 \cI(S^1 \xra{\gamma} M).
	\end{equation}
\end{proposition}

\begin{proof}
	Consider an $n$-simplex $\sigma$ in $X$.
	If $\gamma$ does not intersect $\sigma$ then both sides of \cref{e:identity} applied to the basis element $\sigma$ are $0$.
	Assume that $\gamma$ intersects the faces $d_i \sigma$ and $d_j \sigma$ of $\sigma$.
	By assumption, these are the only faces that it intersects, with $i = j$ an admissible possibility.
	Without loss of generality, let $\gamma_\ast(v)$ be inner-pointing at $d_i \sigma$ and outer-pointing at $d_j \sigma$ with respect to $\sigma$.
\end{proof}

%\section*{Towards and integral relation}
%
%The construction of $W_1$ presented here could be promoted to an integral geometric cocycle by considering (co-)orientations.
%This invites the study of an integral version of the geometric cochain analysis of the Wu relation discussed above.
%
%Let us further assume that $\gamma \colon S^1 \to M$ is an co-oriented immersion.
%For each connected component of $S^1 \setminus W_1$ the fields $f_0, \dots, f_{d-2}$ define a co-orientation of the restriction of $\gamma$ to it.
%Let $U_1 \subseteq S^1 \setminus W_1$ be the union of the closure of the components where this co-orientation agrees with the one chosen for $\gamma \colon S^1 \to M$.
%We remark that the transversality assumption on $\gamma$ ensures that these closed components are disjoint.
%We consider $U_1$ oriented by $v$, and define the integral lift $\widehat{W_1}$ of $W_1$ to be the boundary of $U_1$.
%
%\anibal{Maybe get the integral lift using just co-orientation of $S^1$ since odd and top SW classes are defined with integral coefficients even when $M$ is not orientable.}
